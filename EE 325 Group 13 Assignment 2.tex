\documentclass{article}

\usepackage{fancyhdr}
\usepackage[a4paper, hmargin=1in,vmargin=1.5in]{geometry}
\usepackage{xcolor}
\usepackage{amsmath}
\usepackage{amsthm}
\usepackage{amssymb}
\usepackage{amsfonts}
\usepackage[parfill]{parskip}
\usepackage{hyperref}
\usepackage{url}

\pagestyle{fancy}
\fancyhf{}
\lhead{Assignment 2}
\rhead{Question \thesection}
\lfoot{Group 13}
\rfoot{Page \thepage}
\renewcommand{\footrulewidth}{1pt}
\begin{document}
\title{EE 325 Group 13 - Assignment 2 }
\author{
	Manideep Vudayagiri\\
	\textbf{190070074}
	\and
	Shreyas Nadkarni\\
	\textbf{19D170029}
	\and
	Tarun Kumar\\
	\textbf{19D070062}
	\and
	Ashish Pol\\
	\textbf{190070011}
	\and
	Rohit Ahirwar\\
	\textbf{190070053}
	
	
}

\maketitle
\tableofcontents
\thispagestyle{empty}
\clearpage
\pagenumbering{arabic}

\newpage

\section{Question 1 - (2-5)}
\label{Q1}
\textbf{2-5} Prove and generalize the following identity : \\
\begin{equation*}
	P(A \cup B \cup C) = P(A) + P(B) + P(C) - P(AB) - P(AC) - P(BC) + P(ABC)
\end{equation*}

\hspace{1em} \large{\textbf{SOLUTION :}} \\
We will use the following equation for the proof :
\begin{equation}
\label{(1)}
	P(A \cup B ) = P(A) + P(B) - P(AB)
\end{equation}
\begin{proof}
	Since $A \cup B \cup C = (A \cup B) \cup C$,  using \ref{(1)}, we can say that : \\ 
	\begin{align*}
		    P(A \cup B \cup C) &=P((A \cup B) \cup C) \\
		    &= P(A \cup B) + P(C) - P((A \cup B) \cap C)) &&\text{(using \ref{(1)})} \\
		    &= P(A) + P(B) + P(C) - P(AB) - P((A \cup B) \cap C)) &&\text{(using \ref{(1)})} \\
		    &= P(A) + P(B) + P(C) - P(AB) - P((A \cap C) \cup (B \cap C)) \\ &\text{(using distributive property of $\cap$ and $\cup$)} \\
		    &= P(A) + P(B) + P(C) - P(AB) - \left(P(A \cap C) + P(B \cap C) - P((A \cap C) \cap (B \cap C)) \right) \\ &\text{(using \ref{(1)})} \\
		    &= P(A) + P(B) + P(C) - P(AB) - P(AC) - P(BC) + P(ABC) \\
		    & \text{Since}  (A \cap C) \cap (B \cap C) \equiv A \cap B \cap C  \text{(using distributive property of $\cap$)} 	
	 \end{align*}
\end{proof}

\section{Question 2 - (2-10)}
\label{Q2}
\textbf{2-10} \textit{(Chain rule)} Show that  : \\
\begin{equation*}
	P(A_n \dots A_1) = P(A_n \mid A_{n-1} \dots A_1) \dots P(A_2 \mid A_1)P(A_1)
\end{equation*}

\hspace{1em} \large{\textbf{SOLUTION :}} \\
\textbf{Definition of Conditional Probability :}\\
For events A,B in the same probability space, such that $Pr(B) > 0$, the conditional probability of A given B is : \\
$$P(A \mid B) = \frac{P(A \cap B)}{P(B)} ,
P(A \cap B) = P(A \mid B) \times P(B) $$

 
We shall prove this by induction on $n$ (number of events)\\
So, for the first case $ n = 1 $ \\
$$P(A_1) = P(A_1)$$
Which is trivially true. \\
For the inductive step, let $n > 1 and$ assume (the inductive hypothesis) that : \\
$$P(A_1 \cap A_2 \dots A_{n-1}) = P(A_1) \times P(A_2 \mid A_1) \times \dots P(A_{n-1} \mid (A_1) \cap (A_2) \dots (A_{n-2})) $$ \\
Now we can apply the definition of conditional probability to the two events $A_n$ and $(A_1 \cap A_2 \dots A_{n-1})$ to deduce that : \\
\begin{align*}
	P(A_n \cap A_{n-1} \dots A_1) &= P((A_n) \cap (A_1 \cap A_2 \dots A_{n-1})) \\
	&= P(A_n \mid (A_1 \cap A_2 \dots A_{n-1})) \times P(A_1 \cap A_2 \dots A_{n-1}) \\
	&= P(A_n \mid (A_1 \cap A_2 \dots A_{n-1})) \times P(A_1) \times P(A_2 \mid A_1) \\ &\times \dots P(A_{n-1} \mid (A_1) \cap (A_2) \dots (A_{n-2})) 
\end{align*}
where in the last line we have used the inductive hypothesis. This completes the proof by induction.

\section{Question 3 - (2-14)}
\label{Q3}
\textbf{2-14}  The events $A$ and $B$ are mutually exclusive. Can they be independent?  : \\

\hspace{1em} \large{\textbf{SOLUTION :}} \\

\section{Question 4 - (2-16)}
\label{Q4}
\textbf{2-16}  A box contains $n$ identical balls numbered 1 through $n$. Suppose $k$ balls are drawn in succession. (a) What is the probability that $m$ is the largest number drawn? (b) What is the probability that the largest number drawn is less than or equal to $m$?  : \\

\hspace{1em} \large{\textbf{SOLUTION :}} \\

\section{Question 5 - (2-17)}
\label{Q5}
\textbf{2-17}  Suppose $k$ identical boxes contain $n$ balls numbered 1 through $n$. One ball is drawn from each box. What is the probability that $m$ is the largest number drawn?  : \\

\hspace{1em} \large{\textbf{SOLUTION :}} \\

\section{Question 6 - (2-19)}
\label{Q6}
\textbf{2-19} A box contains $m$ white and $n$ black balls. Suppose k balls are drawn. Find the probability of drawing at least one white ball.  : \\

\hspace{1em} \large{\textbf{SOLUTION :}} \\

\section{Question 7 - (2-24)}
\label{Q7}
\textbf{2-24}  Box 1 contains 1000 bulbs of which 10\% are defective. Box 2 contains 2000 bulbs of which 5\% are defective. Two bulbs are picked from a randomly selected box. (a) Find the probability that both bulbs are defective. (b) Assuming that both are defective, find the probability that they came from box 1.  : \\

\hspace{1em} \large{\textbf{SOLUTION :}} \\

\section{Question 8 - (2-25)}
\label{Q8}
\textbf{2-25} A train and a bus arrive at the station at random between 9 A.M. and 10 A.M. The train stops for 10 minutes and the bus for $x$ minutes. Find $x$ so that the probability that the bus and the train will meet equals 0.5. : \\

\hspace{1em} \large{\textbf{SOLUTION :}} \\

\section{Question 9 - (2-27)}
\label{Q9}
\textbf{2-27} We have two coins; the first is fair and the second two-headed. We pick one of the coins at random, we toss it twice and heads shows both times. Find the probability that the coin picked is fair.  : \\

\hspace{1em} \large{\textbf{SOLUTION :}} \\

\section{Question 10}
\label{Q10}
\textbf{10} A course is taught by four instructors. Before every lecture, the instructors draw lots and one of them is randomly chosen to teach on that day. What is the probability that in N classes, all the lecturers would have taught at least once. Generalise to k instructors teaching the course. : \\

\hspace{1em} \large{\textbf{SOLUTION :}} \\

\section{Question 11}
\label{Q11}
\textbf{11} Numbers $1, 2, 3, 4, 5, and 6$ are randomly placed on a circle. What is the probability that they are placed in increasing order?  : \\

\hspace{1em} \large{\textbf{SOLUTION :}} \\

\section{Question 12}
\label{Q12}
\textbf{12} $A$ and $B$ play the following game of dice. Both roll their dice. If $B$ rolls a one, then it rolls again and keeps whatever appears. The one with the highest value wins. If there is a tie $A$ wins. What is the probability that $A$ wins the game. : \\

\hspace{1em} \large{\textbf{SOLUTION :}} \\

\section{Question 13}
\label{Q13}
\textbf{13}There are $R$ brown balls and $B$ black balls in an urn. Balls are drawn at random without replacement. Let $A_k$ be the event that a brown ball is drawn for the first time on the $k$-th draw. Find $p_k$, the probability of $A_k$. Now consider the case when $B$ and $R$ are increased to $\infty$ while keeping $\alpha = R/(B + R).$ Find $p_k$ as $B + R \rightarrow \infty$. : \\

\hspace{1em} \large{\textbf{SOLUTION :}} \\

\section{Question 14}
\label{Q14}
\textbf{14} There are $n$ of which the $r$-th urn contains $r - 1$ brown balls and $n - r$ black balls. You pick an urn at random and pick two balls at random without replacement. What is the probability that the second ball is black. What is conditional probability that the second ball is black given that the first ball is black.  : \\


\hspace{1em} \large{\textbf{SOLUTION :}} \\

\section{Question 15}
\label{Q15}
\textbf{15} \textbf{Prosecutor’s fallacy} : Let $G$ be the probability that an accused is guilty, and $T$ that the testimony of a witness is true. Many times it is argued that $Prob (G \mid T) = Prob (T \mid G)$. Show that this is true iff $Prob (G) = Prob (T).$ : \\ 


\hspace{1em} \large{\textbf{SOLUTION :}} \\

\section{Question 16}
\label{Q16}
\textbf{16} \textbf{Extra credit}: 10\% of the surface area of a sphere is white and the rest is black. There are no assumptions on how this white part is distributed on the surface. Prove that it is always possible to inscribe a cube with all its vertices black. Think of a randomly inscribed cube. Let $A_i$ be the probability a random vertex is white. Now obtain an upper bound on the probability that at least one of the vertices is white. Show that this strictly less than one. This proves that there is at least one cube with all black vertices.
 : \\

\hspace{1em} \large{\textbf{SOLUTION :}} \\





\end{document}

